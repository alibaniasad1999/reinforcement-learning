

\documentclass{article}
% Template-specific packages

\usepackage{graphicx} % Required for including images
\usepackage[hidelinks]{hyperref}
\usepackage{longtable}
\usepackage{epstopdf}
\usepackage{fullpage,enumitem,amsmath,amssymb,graphicx}
\usepackage{tikz}
\usepackage{pdfpages}
\usepackage{amsmath, amsthm}
\usepackage{fancyhdr}
\usepackage{siunitx}
\usepackage{mathtools}
\usepackage{mathrsfs}
\pagestyle{fancy}
\fancyhead[R]{\rightmark}
\fancyhead[L]{Ali BaniAsad 401209244}
\setlength{\headheight}{10pt}
\setlength{\headsep}{0.2in}
\usepackage{titling}
\usepackage{float}
\newcommand\Tstrut{\rule{0pt}{2.6ex}}         % = `top' strut
\newcommand\Bstrut{\rule[-0.9ex]{0pt}{0pt}}   % = `bottom' strut
%----------------------------------------------------------------------------------------
%	ASSIGNMENT INFORMATION
%----------------------------------------------------------------------------------------



%----------------------------------------------------------------------------------------
\usepackage{graphicx}
\title{Home Work \#4}
\author{Ali BaniAsad 401209244}

\begin{document}
	\maketitle
	\section{Solving Maze with Temporal Difference Learning}
	\subsection{Calculating Value function using TD method}
	In this section, we are going to calculate the value function for the given maze using TD method. The value function is calculated using the following formula:
	\begin{equation}
		V(s) = V(s) + \alpha [r + \gamma V(s') - V(s)]
	\end{equation}
	Where $\alpha$ is the learning rate, $\gamma$ is the discount factor, $r$ is the reward and $s'$ is the next state. The value function is calculated for each state and the results are shown in the following table:
	% add figure
	\begin{figure}[H]
		\centering
		\includegraphics[width=0.7\linewidth]{../figure/value_function_TD}
		\caption{Value function for each state}
		\label{fig:tdvaluefunction}
	\end{figure}

\end{document}

