\section{Question 3}



  Consider the continuing Markov Decision Process (MDP) shown to the right. The only decision to be made is in the top state, where two actions are available: left and right. The numbers indicate the rewards received deterministically after each action. There are exactly two deterministic policies, $\pi_{\text{left}}$ and $\pi_{\text{right}}$. What policy is optimal if $\gamma = 0$? If $\gamma = 0.9$? If $\gamma = 0.5$?

\begin{align*}
\text{State 1:} \quad & r = 0, \quad l = 1 \\
\text{State 2:} \quad & r = 2, \quad l = 0
\end{align*}

The Bellman equation for the optimal action-value function $Q^*(s, a)$ in this case can be expressed as follows:

\[
Q^*(s, a) = R(s, a) + \gamma \cdot \sum_{s'} P(s' | s, a) \cdot \max_{a'} Q^*(s', a')
\]

where $R(s, a)$ represents the immediate reward for taking action $a$ in state $s$, $P(s' | s, a) = 1$ since the transition probabilities are deterministic, and $\gamma$ is the discount factor.

To find the optimal policy for different values of $\gamma$, we can calculate the optimal action-value functions $Q^*(s, a)$ for each state-action pair and determine the actions that maximize these values under the given discount factor.


\begin{itemize}
    \item $\gamma = 0$
    \begin{itemize}
      \item $\pi = \pi_{\text{left}}$
      \[Q^*(s_1, \text{Left}) = 1 + 0 \cdot 0 + 0 \cdot 1 + 0 \cdot 0 + \cdots = 1\]
      \item $\pi = \pi_{\text{right}}$
      \[Q^*(s_1, \text{Right}) = 0 + 0 \cdot 2 + 0 \cdot  + 0 \cdot 0 + \cdots = 0\]
    \end{itemize}
    \item $\gamma = 0.5$
    \begin{itemize}
      \item $\pi = \pi_{\text{left}}$
      \[Q^*(s_1, \text{Left}) = 1 + 0.5 \cdot 0 + 0.5^2 \cdot 1 + 0.5^3 \cdot 0 + \cdots = \sum_{i=0}^{\infty} 0.5^{2i} = \frac{1}{1 - 0.5^2} = \frac{4}{3}\]
      \item $\pi = \pi_{\text{right}}$
      \[Q^*(s_1, \text{Right}) = 0 + 0.5 \cdot 2 + 0.5^2 \cdot 0 + 0.5^3 \cdot 2 + \cdots = \sum_{i=0}^{\infty} 0.5^{2i + 1} = \frac{1}{1 - 0.5^2} \cdot 0.5 \cdot 2= \frac{4}{3}\]
    \end{itemize}
    \item $\gamma = 0.9$
    \begin{itemize}
      \item $\pi = \pi_{\text{left}}$
      \[Q^*(s_1, \text{Left}) = 1 + 0.9 \cdot 0 + 0.9^2 \cdot 1 + 0.9^3 \cdot 0 + \cdots = \sum_{i=0}^{\infty} 0.9^{2i} = \frac{1}{1 - 0.9^2} = \frac{100}{19}\]
      \item $\pi = \pi_{\text{right}}$
      \[Q^*(s_1, \text{Right}) = 0 + 0.9 \cdot 2 + 0.9^2 \cdot 0 + 0.9^3 \cdot 2 + \cdots = \sum_{i=0}^{\infty} 0.9^{2i + 1} = \frac{1}{1 - 0.9^2} \cdot 0.9 \cdot 2= \frac{100}{19} \cdot 0.9 \cdot 2 = \frac{180}{19}\]
      \end{itemize}
\end{itemize}

So, the optimal policy for $\gamma = 0$ is $\pi_{\text{left}}$, the optimal policy for $\gamma = 0.5$ is $\pi_{\text{left}}$ and $\pi_{\text{right}}$, and the optimal policy for $\gamma = 0.9$ is $\pi_{\text{right}}$.

% make tikz picture
\begin{figure}[H]
  \centering
  \caption{Continuing MDP}

\begin{tikzpicture}
  \centering
  % Main circle
  \node [circle, draw, minimum size=1.5cm] (circle1) {};
  
  % Circle 2
  \node [circle, draw, minimum size=1cm, fill=black, below left=1cm and 1cm of circle1] (circle2){};
  
  % Circle 3
  \node [circle, draw, minimum size=1cm, fill=black, below right=1cm and 1cm of circle1] (circle3){};

  % add two more white circles under circle2 and circle3
  \node [circle, draw, minimum size=1.5cm, below=0.5cm of circle2] (circle4){};
  \node [circle, draw, minimum size=1.5cm, below=0.5cm of circle3] (circle5){};
  % add two more black circles under circle4 and circle5
  \node [circle, draw, minimum size=1cm, fill=black, below=0.5cm of circle4] (circle6){};
  \node [circle, draw, minimum size=1cm, fill=black, below=0.5cm of circle5] (circle7){};

  
  % Connect circles
  \draw (circle1) -- (circle2) node[midway, left] {left};
  \draw (circle1) -- (circle3) node[midway, right] {right};
  \draw (circle2) -- (circle4) node[midway, left] {+1};
  \draw (circle3) -- (circle5) node[midway, right] {0};
  \draw (circle4) -- (circle6) node[midway, left] {};
  \draw (circle5) -- (circle7) node[midway, right] {};
  % connect circle6 and circle7 to circle1 curved line
  \draw (circle7) node[pos=60,, right] {+2}to [out=-20,in=40] (circle1);
  \draw (circle6) to [out=-160,in=140] (circle1) node[pos=60, left] {0};

\end{tikzpicture}
\end{figure}



  